\section{Постановка задачи}
\label{sec:Chapter2} \index{Chapter2}

Целью данной работы является проектирование и построение системы автоматической проверки решений студентов к задачам из курсов по программированию.
Система должна иметь свой собственный интерфейс внутри системы LMS.
В нем студенты должны иметь возможность отслеживать назначенные задачи, отправлять решения к ним и просматривать результаты проверок.

Должны поддерживаться разные типы задач для разных курсов.
Многие типы задач уже поддержаны, и их проверка автоматизирована за счет системы Ejudge.
Поэтому новая проектируемая система должна быть интегрирована с системой Ejudge и должна уметь делегировать ей проверку некоторых задач.
Полученный по решению вердикт должен транслироваться в интерфейс LMS, чтобы его могли увидеть преподаватели и студенты.

Одним из альтернативных типов задач могут стать задачи из курса параллельного программирования, для которых отсутствует поддержка в Ejudge. 
В рамках этого курса студенты учатся распараллеливать программы с использованием библиотеки MPI.
Проектируемая система должна уметь самостоятельно компилировать, запускать и проверять такие программы с учетом специфики MPI.
Должно быть поддержано введение дополнительных ограничений на решения студентов.
Так, например, у администратора должна быть возможность выставить ограничение по времени исполнения, по используемой памяти, по количеству использований библиотечных MPI функций.
С точки зрения интерфейса в LMS не должно быть видно никаких отличий между такими задачами и задачами, проверяемыми с помощью Ejudge.

Для достижения заданной цели необходимо выполнить следующие подзадачи: 
\begin{enumerate}

\item
Изучить документацию системы Moodle, на базе которой построена система LMS.
Понять, какие существуют способы расширить ее функционал и спроектировать в ней необходимый интерфейс.

\item
Изучить документацию системы Ejudge.
Понять, каким образом с ней могут взаимодействовать другие программы, чтобы отправлять на проверку решения студентов и получать вердикт по ним.

\item
Спроектировать архитектуру системы автоматической проверки.
Понять, какие модули необходимо реализовать для интеграции уже существующих компонентов.
Спроектированная система должна иметь модульную архитектуру. 
Должен быть механизм добавления новых способов проверки решений студентов без изменения уже существующих.

\item
Спроектировать проверяющий модуль, который будет взаимодействовать с Ejudge и делегировать ему проверку задач.

\item
Спроектировать проверяющий модуль для работы с MPI программами.

\item
Реализовать спроектированную систему и все запланированные модули. 
Развернуть ее во внутренней сети МФТИ.
Провести тестовые контесты с ее использованием.




\end{enumerate}