\section{Обзор области и существующих инструментов}
\label{sec:Chapter3} \index{Chapter3}

На сегодняшний день в мире существует большое число систем автоматизировонного тестирования, многие из которых имеют открытый исходный код,
и которые можно развернуть на своих серверах. 

$PC^{2}$ - система автоматизировонного тестирования, разработанная в Университете штата Калифорния, Сакраменто, США. 
Она активно используется для проверки на международных командных соревнованиях по программированию ICPC. 
Детальное описание системы, включая документацию для администраторов, судей и команд участников, а также все ссылки для скачивания доступны по адресу (\href{https://pc2ccs.github.io/}{https://pc2ccs.github.io/})\cite{pc2github}.
Система разрабатывается с 1988 года, на сегодняшний день последняя стабильная версия 9.7.0 датирована 17 марта 2021 года. 
$PC^{2}$ состоит из клиентской и серверной частей, написанных на языке Java.
Развертывание системы на собственном сервере требует ручных действий и манипуляций с конфигурационными файлами, которые описаны в документации.
Для работы с системой существует графический интерфейс, а также интерфейс командной строки.
Система <<из коробки>> поддерживает множество компиляторов и интерпретаторов. 
Список компиляторов может быть вручную расширен администратором системы.
Для интерпретаторов такой возможности нет, однако в список включены PERL, PHP, Ruby, Python и shell.

Yandex.Contest (\href{https://contest.yandex.ru/}{https://contest.yandex.ru/}) \cite{YCwebsite} - сервис для онлайн проверки заданий, ориентированный в основном на олимпиады и соревнования по программированию.
На базе этой системы проводится ежегодный чемпионат по программированию от Яндекса, а также различные олимпиады, в том числе финальные этапы всероссийской олимпиады по программированию. 
Yandex.Contest поддерживает более двадцати языков программирования и позволяет администраторам гибко настраивать схему соревнований. 
Также гибко можно настраивать задания: постановку задачи, набор тестов, критерии оценки. 
Утверждается, что сервис способен одновременно обрабатывать терабайты данных, благодаря чему способен выдержать одновременную нагрузку более чем от тысячи участников. 

DOMjudge - система автоматизировонного тестирования, разработанная Исследовательской Ассоциацией A-Eskwadraat Утрехтского университета, Нидерланды.
Система поставляется в виде серверного ПО и клиентского web-интерфейса.
Основным преимуществом данной системы является возможность легко настраивать процесс параллельной обработки несколькими серверами.
Это дает возможность горизонтально масштабировать систему при растущей нагрузке. 
Проверка может осуществляться на серверах из внешней сети.
То есть можно использовать облачные решения для развертывания системы. 
Установка и настройка системы происходит с помощью скриптов. 
Имеется возможность конфигурировать систему через ручное редактирование файлов. 
Однако стоит отметить, что <<из коробки>> поддерживается не так много языков: C, C++, Java, Haskell.


