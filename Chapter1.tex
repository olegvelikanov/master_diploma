\section{Введение}
\label{sec:Chapter1} \index{Chapter1}

В наши дни большинство университетских курсов по программированию содержат большой блок практической (лабораторной) работы, 
которая подразумевает, что студенты решают задачи по программированию: пишут программы, реализуют алгоритмы. 
Эти программы как и решения практических задач по любому курсу необходимо проверять преподавателям.
При этом решение одной и той же задачи у каждого студента будет уникальным, каждый студент имеет свой стиль написания кода.
Поэтому ручная проверка таких задач может отнимать много времени. 
Такая проверка является довольно рутинным процессом, в которых человеку свойственно ошибаться.
Хотелось бы иметь возможность как-то автоматизировать этот процесс.

В современных практиках промышленной разработки давно закрепился подход юнит-тестирования. 
Программу делят на составные части (юниты).
В качестве таких частей могут выступать разные средства языка, например функции или классы.
Дальше к каждой части формируется свой набор тестовых сценариев.
Если говорить о тестировании функций, то тестовый сценарий будет состоять из входных аргументов функции и значения, 
которое мы ожидаем получить на выходе функции, если применить ее к заданным аргументам.
Чтобы протестировать функцию, надо запустить ее на всех наборах аргументов и проверить равенство ожидаемых и полученных при запуске значений.

Программы студентов в рамках учебных курсов решают достаточно узкие задачи: 
делают какое-то преобразование над входными данными, вычисляют какое-то значение на их основе или реализуют какой-то изучаемый алгоритм над ними.
К проверке таких задач можно применить подход юнит-тестирования.
Это позволит снять с преподавателей значительную часть работы. 
Позволит избавиться от ограничения на количество задач, которое может успеть проверить преподаватель.
Можно будет увеличить число задач, предлагаемых студентам на практических занятиях.
Студент будет сразу получать оценку своей работы.
Поэтому такой подход к проверке давно получил популярность и широко используется в университетах. 

При использовании этой методики сдача лабораторной работы выглядит следующим образом. 
Студент отправляет решение в систему автоматизированной проверки.
Система запускает программу с разными входными данными, сверяет выходные данные с заранее известным эталоном, выносит вердикт.
На решение задачи может быть дано несколько попыток, засчитывается последнее отправленное решение. 
По итогу формируется оценка, полученная исключительно автоматическим тестированием.
После окончания лабораторной преподаватель может в ручном режиме посмотреть какие-то решения и скорректировать оценку. 

В МФТИ активно используется система организации и проверки контестов - Ejudge. 
В ее терминах контест это набор задач, на решение которых отводится какое-то конечное время.
Система имеет открытый исходный код, поддерживает большой набор языков программирования, имеет большой инструментарий для администрирования.
На данный момент она развернута во внутренней сети МФТИ, и практические занятия большинства курсов организованы с ее помощью.
С другой стороны в последнее время в институте активно развивается своя собственная система управления учебным процессом - LMS (learning management system),
построенная на базе Moodle.
Она обеспечивает связь преподавателя и студента, 
позволяет адресно предоставить студентам доступ к учебным материалам самого разнообразного характера,
позволяет принимать от студентов и удобно проверять различные типы заданий. 
В связи с чем появилась идея интеграции систем LMS и Ejudge.  

Хотелось бы дать возможность студентам видеть назначенные контесты, отправлять решение задач, отслеживать свою успеваемость 
в рамках курсов по программированию в том же интерфейсе, который они используют для курсов других кафедр. 
Преподавателям это даст возможность в одном в интерфейсе поддерживать контакт со студентами, назначать им задания, 
управлять задачами для практических занятий, видеть успеваемость студентов по конкретному контесту и по всему курсу в целом. 
Чтобы реализовать такую интеграцию необходимо построить промежуточную систему, которая будет выполнять роль посредника, 
способного общаться в терминах обеих систем. 

Несмотря на то, что система организации контестов Ejudge имеет широкий спектр поддерживаемых языков,  
с ее помощью невозможно автоматизировать проверку задач по курсу параллельного программирования.
В рамка этого курса студенты изучают подходы к распараллеливанию классических алгоритмов вычислительной математики. 
В качестве языка используется \textbf{\emph{C}} вместе с библиотекой MPI.
В Ejudge нет поддержки компиляции с использованием утилиты \textbf{\emph{mpicc}}, запуска программ с использованием \textbf{\emph{mpirun}}.
Ejudge умеет запускать один процесс и сравнивать его stdout с эталоном. 
Поэтому невозможно независимо проверять выходные данные отдельных процессов. 
Давно есть запрос на автоматизацию проверки задач этого курса, и сделать это в рамках Ejudge не представляется возможным.

Учитывая все эти предпосылки, появляется необходимость построить новую систему, которая предоставляла бы единый интерфейс для проверки самых разных задач, 
при этом внутри себя реализовывала бы разные механизмы проверки.
Одним из таких механизмов была бы делегация проверки в Ejudge, а одним из альтернативных способов проверки стала бы проверка MPI программ, 
которую необходимо реализовать с нуля.