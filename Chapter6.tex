\section{Заключение}
\label{sec:Chapter6} \index{Chapter6}
Целью работы было построение системы автоматической проверки студенческих программ.
Система должна была иметь интерфейс внутри системы дистаницонного обучения LMS, используемой в МФТИ.
В ходе работы изучена документация проекта Moodle, на базе которого построена LMS.
Стали известны методы для расширения ее функционала и создания нового интрфейса сдачи решения задач на проверку.
Оказалось, что Moodle позволяет реализовать собственные плагины, которые будут добавлять новый функционал.

Затем была изучена документация Ejudge.
Основной задачей было понять, какие существуют способы программного взаимодействовия с системой проверки.
Выяснилось, что система имеет интерфейс командной строки, который в частности используется вэб сервисом 
для обработки пользовательских запросов, получемых через вэб интерфейс.
В итоге был построен модуль проверки посредством использования Ejudge, который взаимодействует с Ejudge через этот интерфейс.

Далее была спроектирована архитектура системы автоматической проверки (рис. \ref{fig:design}).
При разработке архитетурной схемы были учтены все требования, предъявляемые к системе: 
наличие REST API для отправки решения, поддержка нескольких способов проверки и возможность добавлять новые.
Появился список сервисов, которые необходимо реализовать.
Также стал понятен список внешних технологий (база данных, брокер сообщений),
которые необходимо выбрать из множества доступных и настроить для использования.
В итоге в рамках проекта были использованы база данных PostgreSQL и брокер сообщений RabbitMQ.

Последним шагом была реализация спроектированных сервисов.
Для этого было выбран язык Python как наиболее простой для понимания и широко распространненой в академической среде.
Были реализованы:
\begin{enumerate}
    \item сервис \emph{coordinator}, который предоставляет REST API для отправки решения на проверку и получения актуального статуса проверки.
    \item проверяющий модуль \emph{ejudge-checker}, который делегирует проверку задач системе Ejudge
    \item проверяющий модуль \emph{mpi-checker}, который самостоятельно осущестляет проверку MPI задач
\end{enumerate}
Стоит отметить, что для реализации всех требований, предъявляемых к проверке MPI задач, дополнительно был разработан MPI профайлер.

Таким образом, все поставленные в разделе [\ref{sec:Chapter2}] задачи выполнены и цели данной работы достигнуты. 